\documentclass[12pt]{article}
\usepackage[paper=letterpaper,margin=1.5cm]{geometry}
\usepackage{amsmath}
\usepackage{amssymb}
\usepackage{amsfonts}
\usepackage{newtxtext, newtxmath}
\usepackage{enumitem}
\usepackage{titling}
\usepackage{svg}
\usepackage{xcolor}
\usepackage{listings}
\usepackage{float}
\usepackage{nicefrac}
\usepackage{multirow}
\usepackage[most]{tcolorbox}
\usepackage[colorlinks=true]{hyperref}

\setlength{\droptitle}{-6em}

\definecolor{codegreen}{rgb}{0,0.6,0}
\definecolor{codegray}{rgb}{0.5,0.5,0.5}
\definecolor{codepurple}{rgb}{0.58,0,0.82}
\definecolor{backcolour}{rgb}{0.95,0.95,0.92}
\definecolor{bg}{rgb}{1,0.96,0.9}

\lstdefinestyle{mystyle}{
  commentstyle=\color{codegreen},
  keywordstyle=\color{magenta},
  numberstyle=\tiny\color{codegray},
  stringstyle=\color{codepurple},
  basicstyle=\ttfamily\footnotesize,
  breakatwhitespace=false,
  breaklines=true,
  captionpos=b,
  keepspaces=true,
  numbers=left,
  numbersep=5pt,
  showspaces=false,
  showstringspaces=false,
  showtabs=false,
  tabsize=2
}

% \tcbset{enlarge left by=-0.8cm,left=1.2cm,enlarge right by=-2cm,right=0.8cm}

\lstset{
  style=mystyle,
  inputencoding=utf8,
  extendedchars=true,
}

\newcommand{\question}[1]{\begin{tcolorbox}[enhanced jigsaw,colback=bg,boxrule=0pt,arc=1pt,halign=center] #1 \end{tcolorbox}}

\begin{document}

\begin{enumerate}[leftmargin=\labelsep]
  \question {
  \item Considering the following two-dimensional measurements:

        \begin{table}[H]
          \centering
          \begin{tabular}{c|c|c}
                  & $y_1$ & $y_2$ \\ \hline
            $x_1$ & -2    & 2     \\
            $x_2$ & -1    & 3     \\
            $x_3$ & 0     & 1     \\
            $x_4$ & -2    & 1
          \end{tabular}
        \end{table}

        \begin{enumerate}
          \item What are the maximum likelihood parameters of a multivariate Gaussian
                distribution for this data set?
          \item What is the Gaussian's shape? Draw its contour plot.
        \end{enumerate}
        }

        \begin{enumerate}
          \item {
                Regarding the motivation for finding the "maximum likelihood parameters",
                \href{https://en.wikipedia.org/wiki/Maximum_likelihood_estimation}{this} and
                other articles are helpful to delineate the MLE method, which aims
                to find the parameters of a distribution that maximize the probability
                of observing a given data set. Considering a data set normally distributed
                (as referenced in the question's statement), the MLE method is equivalent
                to finding the parameters of the distribution that minimize the sum of
                squared errors between the data set and the distribution's probability
                density function.

                As we know, a multivariate Gaussian distribution is defined by its mean vector
                $\boldsymbol{\mu}$ and its covariance matrix $\boldsymbol{\Sigma}$:

                $$
                  \mathcal{N}(\boldsymbol{x} \mid \boldsymbol{\mu}, \boldsymbol{\Sigma}) =
                  \frac{1}{(2\pi)^{d/2}|\boldsymbol{\Sigma}|^{1/2}}
                  \exp\left(-\frac{1}{2}(\boldsymbol{x} - \boldsymbol{\mu})^T
                  \boldsymbol{\Sigma}^{-1}(\boldsymbol{x} - \boldsymbol{\mu})\right)
                $$

                The mean vector is simply the average of the data points (considering each coordinate):

                $$
                  \boldsymbol{\mu} = \frac{1}{n} \sum_{i=1}^n \boldsymbol{x}_i
                $$

                The covariance matrix, on the other hand, requires a bit more work,
                requiring us to calculate the deviations of each data point from the mean:

                $$
                  \boldsymbol{\Sigma} = \frac{1}{n - 1} \sum_{i=1}^n (\boldsymbol{x}_i - \boldsymbol{\mu}) (\boldsymbol{x}_i - \boldsymbol{\mu})^T
                $$

                Note, of course, that $\boldsymbol{\Sigma}$ is a symmetric matrix.
                We can, then, start calculating the maximum likelihood parameters:

                $$
                  \boldsymbol{\mu} = \frac{1}{4} \begin{bmatrix}
                    -2 - 1 + 0 - 2 \\
                    2 + 3 + 1 + 1
                  \end{bmatrix} = \begin{bmatrix}
  -1.25\\
  1.75\\
\end{bmatrix}
                $$

                $$
                  \boldsymbol{\Sigma} = \frac{1}{3} \begin{bmatrix}
                    (-2 + 1.25)^2 + \cdots + (-2 + 1.25)^2 & (-2 + 1.25)(2 - 1.75) + \cdots       \\
                    (-2 + 1.25)(2 - 1.75) + \cdots         & (2 - 1.75)^2 + \cdots + (1 - 1.75)^2
                  \end{bmatrix} = \begin{bmatrix}
  6.66667 & 2.66667\\
  2.66667 & 2\\
\end{bmatrix}
                $$

                As such, we can calculate both the determinant and the inverse of the covariance matrix,
                effectively gathering all the needed parameters for the multivariate Gaussian distribution:

                $$
                  |\boldsymbol{\Sigma}| = 0.916667^2 - 0.083333^2 = 0.8(3)
                $$

                $$
                  X = \begin{bmatrix}
                    a & b \\
                    c & d
                  \end{bmatrix} \implies X^{-1} = \frac{1}{ac - bd} \begin{bmatrix}
                    d  & -b \\
                    -c & a
                  \end{bmatrix}; \quad
                  \boldsymbol{\Sigma}^{-1} = \frac{1}{0.8(3)} \begin{bmatrix}
                    0.916667 & 0.083333 \\
                    0.083333 & 0.916667
                  \end{bmatrix} = \begin{bmatrix}
  5 & 0\\
  0 & 5\\
\end{bmatrix}
                $$
                }
          \item {
                The following contour plot was computed utilizing Python (the code snippets
                are available in this sheet's respective notebook):

                \begin{figure}[H]
                  \centering
                  \includesvg[width=0.5\textwidth]{assets/ex-1/contour.svg}
                \end{figure}

                Contour plots are a great way to visualize the shape of a multivariate Gaussian
                distribution in two dimensions. The plot above shows the distribution's
                probability density function, which is proportional to the probability
                of observing a data point in a given region - the darker the region, the
                higher the probability of observing a data point in that region!

                Essentially, each line in the plot above represents a constant probability
                density value. The ellipsoid is always centered at the mean vector,
                and its shape is determined by the covariance matrix.
                The ellipsoid's axes' directions are determined by the eigenvectors of the
                covariance matrix, and its axes' lengths are determined by the square root
                of the eigenvalues of the covariance matrix. The ellipsoid's axes' lengths
                are, then, proportional to the standard deviations of the distribution along each
                axis.
                }

        \end{enumerate}

        \question {
  \item Consider the following data-set, paired with a query vector $x_{new} = \begin{bmatrix} 1 & 1 & 1 & 1 & 1\end{bmatrix}^T$:

        \begin{table}[H]
          \centering
          \begin{tabular}{c|c|c|c|c|c|c}
                  & $y_1$ & $y_2$ & $y_3$ & $y_4$ & $y_5$ & $z$ \\ \hline
            $x_1$ & 1     & 1     & 0     & 1     & 0     & 1   \\
            $x_2$ & 1     & 1     & 1     & 0     & 0     & 0   \\
            $x_3$ & 0     & 1     & 1     & 1     & 0     & 0   \\
            $x_4$ & 0     & 0     & 0     & 1     & 1     & 0   \\
            $x_5$ & 1     & 0     & 1     & 1     & 1     & 1   \\
            $x_6$ & 0     & 0     & 1     & 0     & 0     & 1   \\
            $x_7$ & 0     & 0     & 0     & 0     & 1     & 1   \\
          \end{tabular}
        \end{table}

        \begin{enumerate}
          \item Using Bayes' rule, without making any assumptions, compute the posterior
                probabilities for the query vector. How is it classified?
          \item What is the problem of working without assumptions?
          \item Compute the class for the same query vector under the naive Bayes assumption?
          \item Consider the presence of missings. Under the same naive Bayes assumption,
                how would you classify the query vector $x_{new} = \begin{bmatrix} 1 & ? & 1 & ? & 1\end{bmatrix}^T$?
        \end{enumerate}
        }

        \begin{enumerate}
          \item {}
          \item {}
          \item {}
          \item {}
        \end{enumerate}

        \question {
  \item Considering the following data set, paired with the query vector $x_{new} = \begin{bmatrix} 100 & 225 \end{bmatrix}^T$:

        \begin{table}[H]
          \centering
          \begin{tabular}{c|c|c|c}
                  & $y_1$ & $y_2$ & z \\ \hline
            $x_1$ & 170   & 160   & 0 \\
            $x_2$ & 80    & 220   & 1 \\
            $x_3$ & 90    & 200   & 1 \\
            $x_4$ & 60    & 160   & 0 \\
            $x_5$ & 50    & 150   & 0 \\
            $x_6$ & 70    & 190   & 1
          \end{tabular}
        \end{table}

        \begin{enumerate}
          \item Compute the most probable class for the query vector assuming that the likelihoods are 2-dimensional Gaussians.
          \item Compute the most probable class for the query vector, under the Naive Bayes assumption,
                using 1-dimensional Gaussians to model the likelihoods.
        \end{enumerate}
        }

        \begin{enumerate}
          \item {}
          \item {}
        \end{enumerate}

        \question {
  \item Assuming training examples with $m$ features and a binary class:

        \begin{enumerate}
          \item How many parameters are needed to estimate, considering boolean features and:
                \begin{enumerate}
                  \item No assumptions regarding the data's distribution.
                  \item Naive Bayes assumption.
                \end{enumerate}
          \item How many parameters are needed to estimate, considering numeric-valued features and:
                \begin{enumerate}
                  \item Multivariate Gaussian assumption.
                  \item Naive Bayes assumption.
                \end{enumerate}
        \end{enumerate}
        }

        \begin{enumerate}
          \item {
                \begin{enumerate}
                  \item {}
                  \item {}
                \end{enumerate}
                }
          \item {
                \begin{enumerate}
                  \item {}
                  \item {}
                \end{enumerate}
                }
        \end{enumerate}

\end{enumerate}

\end{document}