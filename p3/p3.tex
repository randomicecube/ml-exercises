\documentclass[12pt]{article}
\usepackage[paper=letterpaper,margin=1.5cm]{geometry}
\usepackage{amsmath}
\usepackage{amssymb}
\usepackage{amsfonts}
\usepackage{newtxtext, newtxmath}
\usepackage{enumitem}
\usepackage{titling}
\usepackage{svg}
\usepackage{xcolor}
\usepackage{listings}
\usepackage{float}
\usepackage{nicefrac}
\usepackage{multirow}
\usepackage[most]{tcolorbox}
\usepackage[colorlinks=true]{hyperref}

\setlength{\droptitle}{-6em}

\definecolor{codegreen}{rgb}{0,0.6,0}
\definecolor{codegray}{rgb}{0.5,0.5,0.5}
\definecolor{codepurple}{rgb}{0.58,0,0.82}
\definecolor{backcolour}{rgb}{0.95,0.95,0.92}
\definecolor{bg}{rgb}{1,0.96,0.9}

\lstdefinestyle{mystyle}{
  commentstyle=\color{codegreen},
  keywordstyle=\color{magenta},
  numberstyle=\tiny\color{codegray},
  stringstyle=\color{codepurple},
  basicstyle=\ttfamily\footnotesize,
  breakatwhitespace=false,
  breaklines=true,
  captionpos=b,
  keepspaces=true,
  numbers=left,
  numbersep=5pt,
  showspaces=false,
  showstringspaces=false,
  showtabs=false,
  tabsize=2
}

% \tcbset{enlarge left by=-0.8cm,left=1.2cm,enlarge right by=-2cm,right=0.8cm}

\lstset{
  style=mystyle,
  inputencoding=utf8,
  extendedchars=true,
}

\newcommand{\question}[1]{\begin{tcolorbox}[enhanced jigsaw,colback=bg,boxrule=0pt,arc=1pt,halign=center] #1 \end{tcolorbox}}

\begin{document}

\begin{enumerate}[leftmargin=\labelsep]
  \question {
  \item Considering the following two-dimensional measurements:

        \begin{table}[H]
          \centering
          \begin{tabular}{c|c|c}
                  & $y_1$ & $y_2$ \\ \hline
            $x_1$ & -2    & 2     \\
            $x_2$ & -1    & 3     \\
            $x_3$ & 0     & 1     \\
            $x_4$ & -2    & 1
          \end{tabular}
        \end{table}

        \begin{enumerate}
          \item What are the maximum likelihood parameters of a multivariate Gaussian
                distribution for this data set?
          \item What is the Gaussian's shape? Draw its contour plot.
        \end{enumerate}
        }

        \begin{enumerate}
          \item {
                Regarding the motivation for finding the "maximum likelihood parameters",
                \href{https://en.wikipedia.org/wiki/Maximum_likelihood_estimation}{this} and
                other articles are helpful to delineate the MLE method, which aims
                to find the parameters of a distribution that maximize the probability
                of observing a given data set. Considering a data set normally distributed
                (as referenced in the question's statement), the MLE method is equivalent
                to finding the parameters of the distribution that minimize the sum of
                squared errors between the data set and the distribution's probability
                density function.

                As we know, a multivariate Gaussian distribution is defined by its mean vector
                $\boldsymbol{\mu}$ and its covariance matrix $\boldsymbol{\Sigma}$:

                $$
                  \mathcal{N}(\boldsymbol{x} \mid \boldsymbol{\mu}, \boldsymbol{\Sigma}) =
                  \frac{1}{(2\pi)^{d/2}|\boldsymbol{\Sigma}|^{1/2}}
                  \exp\left(-\frac{1}{2}(\boldsymbol{x} - \boldsymbol{\mu})^T
                  \boldsymbol{\Sigma}^{-1}(\boldsymbol{x} - \boldsymbol{\mu})\right)
                $$

                The mean vector is simply the average of the data points (considering each coordinate),
                while the covariance matrix, on the other hand, requires a bit more work,
                requiring us to calculate the deviations of each data point from the mean:

                $$
                  \boldsymbol{\mu} = \frac{1}{n} \sum_{i=1}^n \boldsymbol{x}_i, \quad
                  \boldsymbol{\Sigma} = \frac{1}{n - 1} \sum_{i=1}^n (\boldsymbol{x}_i - \boldsymbol{\mu}) (\boldsymbol{x}_i - \boldsymbol{\mu})^T
                $$

                Note, of course, that $\boldsymbol{\Sigma}$ is a symmetric matrix.
                We can, then, start calculating the maximum likelihood parameters:

                $$
                  \boldsymbol{\mu} = \frac{1}{4} \begin{bmatrix}
                    -2 - 1 + 0 - 2 \\
                    2 + 3 + 1 + 1
                  \end{bmatrix} = \begin{bmatrix}
  -1.25\\
  1.75\\
\end{bmatrix}
                $$

                $$
                  \boldsymbol{\Sigma} = \frac{1}{3} \begin{bmatrix}
                    (-2 + 1.25)^2 + \cdots + (-2 + 1.25)^2 & (-2 + 1.25)(2 - 1.75) + \cdots       \\
                    (-2 + 1.25)(2 - 1.75) + \cdots         & (2 - 1.75)^2 + \cdots + (1 - 1.75)^2
                  \end{bmatrix} = \begin{bmatrix}
  6.66667 & 2.66667\\
  2.66667 & 2\\
\end{bmatrix}
                $$

                As such, we can calculate both the determinant and the inverse of the covariance matrix,
                effectively gathering all the needed parameters for the multivariate Gaussian distribution:

                $$
                  |\boldsymbol{\Sigma}| = 0.916667^2 - 0.083333^2 = 0.8(3)
                $$

                $$
                  X = \begin{bmatrix}
                    a & b \\
                    c & d
                  \end{bmatrix} \implies X^{-1} = \frac{1}{ac - bd} \begin{bmatrix}
                    d  & -b \\
                    -c & a
                  \end{bmatrix}; \quad
                  \boldsymbol{\Sigma}^{-1} = \frac{1}{0.8(3)} \begin{bmatrix}
                    0.916667 & 0.083333 \\
                    0.083333 & 0.916667
                  \end{bmatrix} = \begin{bmatrix}
  5 & 0\\
  0 & 5\\
\end{bmatrix}
                $$
                }
          \item {
                The following contour plot was computed utilizing Python (the code snippets
                are available in this sheet's respective notebook):

                \begin{figure}[H]
                  \centering
                  \includesvg[width=0.45\textwidth]{assets/ex-1/contour.svg}
                \end{figure}

                Contour plots are a great way to visualize the shape of a multivariate Gaussian
                distribution in two dimensions. The plot above shows the distribution's
                probability density function, which is proportional to the probability
                of observing a data point in a given region - the darker the region, the
                higher the probability of observing a data point in that region!

                Essentially, each line in the plot above represents a constant probability
                density value. The ellipsoid is always centered at the mean vector,
                and its shape is determined by the covariance matrix.
                The ellipsoid's axes' directions are determined by the eigenvectors of the
                covariance matrix, and its axes' lengths are determined by the square root
                of the eigenvalues of the covariance matrix. The ellipsoid's axes' lengths
                are, then, proportional to the standard deviations of the distribution along each
                axis.
                }

        \end{enumerate}

        \question {
  \item Consider the following data-set, paired with a query vector $x_{new} = \begin{bmatrix} 1 & 1 & 1 & 1 & 1\end{bmatrix}^T$:

        \begin{table}[H]
          \centering
          \begin{tabular}{c|c|c|c|c|c|c}
                  & $y_1$ & $y_2$ & $y_3$ & $y_4$ & $y_5$ & $z$ \\ \hline
            $x_1$ & 1     & 1     & 0     & 1     & 0     & 1   \\
            $x_2$ & 1     & 1     & 1     & 0     & 0     & 0   \\
            $x_3$ & 0     & 1     & 1     & 1     & 0     & 0   \\
            $x_4$ & 0     & 0     & 0     & 1     & 1     & 0   \\
            $x_5$ & 1     & 0     & 1     & 1     & 1     & 1   \\
            $x_6$ & 0     & 0     & 1     & 0     & 0     & 1   \\
            $x_7$ & 0     & 0     & 0     & 0     & 1     & 1   \\
          \end{tabular}
        \end{table}

        \begin{enumerate}
          \item Using Bayes' rule, without making any assumptions, compute the posterior
                probabilities for the query vector. How is it classified?
          \item What is the problem of working without assumptions?
          \item Compute the class for the same query vector under the naive Bayes assumption?
          \item Consider the presence of missings. Under the same naive Bayes assumption,
                how would you classify the query vector $x_{new} = \begin{bmatrix} 1 & ? & 1 & ? & 1\end{bmatrix}^T$?
        \end{enumerate}
        }

        \begin{enumerate}
          \item {
                Bayes' rule tells us that the posterior probability of a class $c$ given a query vector $x_{new}$ is:

                $$
                  P(z = c | x_{new}) = \frac{P(x_{new} | z = c) P(z = c)}{P(x_{new})}
                $$

                We call \textbf{priors} the probabilities $P(z = c)$, and \textbf{likelihoods} the "probabilities" $P(x_{new} | z = c)$.
                To translate the product between the likelihood and the prior into an actual
                probability, we need to normalize such product by the probability $P(x_{new})$,
                which is the probability of observing the query vector $x_{new}$ "in the wild", regardless of the class.

                Here, we have two possible classes (i.e, $z$ values) possible for
                any given query vector: $z = 0$ and $z = 1$. Note that we're not
                allowed to make any assumptions - that is, we mustn't assume any
                conditional independence or distribution for the data's features.

                By looking at the data set, we can deduce the \textbf{priors} for each class:

                $$
                  P(z = 0) = \frac{3}{7} \quad \text{and} \quad P(z = 1) = \frac{4}{7}
                $$

                Without any assumptions, however, $P(x_{new})$ will essentially be the
                probability of observing the query vector $x_{new}$ in the "training" data set.
                Since we never observed the query vector $x_{new}$ in the training data set,
                $P(x_{new})$ is zero (as are the likelihoods). As such, we can't use Bayes' rule to compute the
                posterior probabilities for the query vector $x_{new}$, and such a sample
                would be \textbf{unclassified}.

                }
          \item {
                Without assumptions, we're essentially strapped to the data set we have at hand.
                In cases like the one at hand, for example, we're unable to compute the posterior
                probabilities for the query vector $x_{new}$ (and for a myriad of other query vectors).
                As such, data sets with a large number of features and/or a small number of samples
                are problematic to work with without making any assumptions, since we're
                more likely than not going to run into this problem often.
                }
          \item {
                The Naive Bayes assumption essentially states that the features are conditionally
                independent given the class. In other words, it assumes that observing a given
                feature in the presence of a class is completely independent of observing any other feature.
                Such a strong assumption, although often misleading, is often a good starting point for
                classification problems, letting us compute more easily the posterior probabilities
                for new query vectors. Regarding $x_{new}$, we can compute the posterior probabilities
                as follows (and considering the previously computed priors):

                $$
                  P(z = c | x_{new}) = \frac{P(x_{new} | z = c) P(z = c)}{P(x_{new})}
                  = \frac{P(y_1 = 1 \mid z = c) \cdot ... \cdot P(y_5 = 1 \mid z = c) \cdot P(z = c)}{P(y_1 = 1) \cdot ... \cdot P(y_5 = 1)}
                $$

                The denominator will always be the same for both classes, so we can ignore it,
                effectively needing only to compute the numerator for each class. The numerators
                for each class are given by:

                \begin{align*}
                  z = 0: \quad (\nicefrac{1}{3})^2 \cdot (\nicefrac{2}{3})^3 \cdot (\nicefrac{3}{7}) \approx 0.014109 \\
                  z = 1: \quad (\nicefrac{2}{4})^4 \cdot (\nicefrac{1}{4})^1 \cdot (\nicefrac{4}{7}) \approx 0.0089
                \end{align*}

                With class $z=0$ having a higher posterior probability, the query vector $x_{new}$
                is classified as belonging to class $z=0$.
                }
          \item {
                Here, we're essentially asking for the probability of observing a given sample
                with features $y_1$, $y_3$ and $y_5$ valued at $1$, considering that the
                feature values for both $y_2$ and $y_4$ are missing (i.e., unknown). We can
                compute the posterior probabilities for the query vector $x_{new}$ as follows:

                $$
                  P(z = c | x_{new})
                  = \frac{P(y_1 = 1 \mid z = c) \cdot P(y_3 = 1 \mid z = c) \cdot P(y_5 = 1 \mid z = c) \cdot P(z = c)}{P(y_1 = 1) \cdot P(y_3 = 1) \cdot P(y_5 = 1)}
                $$

                The denominator will always be the same for both classes, so we can ignore it,
                once again. The numerators for each class are given by:

                \begin{align*}
                  z = 0: \quad (\nicefrac{1}{3})^2 \cdot (\nicefrac{2}{3}) \cdot (\nicefrac{3}{7}) \approx 0.031746 \\
                  z = 1: \quad (\nicefrac{2}{4})^3 \cdot (\nicefrac{4}{7}) \approx 0.07143
                \end{align*}

                With class $z=1$ having a higher posterior probability, the query vector $x_{new}$,
                with missing values for $y_2$ and $y_4$, is classified as belonging to class $z=1$.
                }
        \end{enumerate}

        \question {
  \item Considering the following data set, paired with the query vector $x_{new} = \begin{bmatrix} 100 & 225 \end{bmatrix}^T$:

        \begin{table}[H]
          \centering
          \begin{tabular}{c|c|c|c}
                  & $y_1$ & $y_2$ & z \\ \hline
            $x_1$ & 170   & 160   & 0 \\
            $x_2$ & 80    & 220   & 1 \\
            $x_3$ & 90    & 200   & 1 \\
            $x_4$ & 60    & 160   & 0 \\
            $x_5$ & 50    & 150   & 0 \\
            $x_6$ & 70    & 190   & 1
          \end{tabular}
        \end{table}

        \begin{enumerate}
          \item Compute the most probable class for the query vector, assuming that the likelihoods are 2-dimensional Gaussians.
          \item Compute the most probable class for the query vector, under the Naive Bayes assumption,
                using 1-dimensional Gaussians to model the likelihoods.
        \end{enumerate}
        }

        \begin{enumerate}
          \item {
                Just as had been illustrated in the previous question, Bayes' rule tells
                us that the probability of a class given a query vector is proportional to
                the product of the likelihood and the prior. As such, we can compute the
                posterior probabilities for the query vector $x_{new}$ as follows:

                $$
                  P(z = c | x_{new}) = \frac{P(x_{new} | z = c) P(z = c)}{P(x_{new})}
                $$

                The priors are calculated in the same way as before:

                $$
                  P(z = 0) = P(z = 1) = \nicefrac{3}{6} = \nicefrac{1}{2}
                $$

                Here, however, our likelihoods are 2-dimensional Gaussians:

                $$
                  P(x \mid z = c) \sim
                  \mathcal{N}(\boldsymbol{x} \mid \boldsymbol{\mu_c}, \boldsymbol{\Sigma_c}) =
                  \frac{1}{(2\pi)^{d/2}|\boldsymbol{\Sigma_c}|^{1/2}}
                  \exp\left(-\frac{1}{2}(\boldsymbol{x} - \boldsymbol{\mu_c})^T
                  \boldsymbol{\Sigma_c}^{-1}(\boldsymbol{x} - \boldsymbol{\mu_c})\right)
                $$

                Both the mean and covariance matrix are relatively straight forward
                to compute; having done so in the past, we'll simply plug the
                results here (the Python code used to compute the results is
                available in the notebook):

                \begin{align*}
                  \boldsymbol{\mu_0} & = \begin{bmatrix}
  93.3333\\
  156.667\\
\end{bmatrix} \quad
                                     &                                            & \boldsymbol{\Sigma_0} = \begin{bmatrix}
  4433.33 & 216.667\\
  216.667 & 33.3333\\
\end{bmatrix} \\
                  \boldsymbol{\mu_1} & = \begin{bmatrix}
  80\\
  203.333\\
\end{bmatrix} \quad
                                     &                                            & \boldsymbol{\Sigma_1} = \begin{bmatrix}
  100 & 50\\
  50 & 233.333\\
\end{bmatrix}
                \end{align*}

                Considering that, for both classes, both the denominator and the prior
                values will be the same, it'll be the likelihoods defining whether
                we assign the query vector to class $z=0$ or $z=1$. The likelihoods
                for each class are calculated as follows:

                \begin{align*}
                  P(x_{new} \mid z = 0) \sim \mathcal{N}(x_{new} \mid \mu_0, \Sigma_0)
                   & = \frac{1}{(2\pi)^{2/2}|\Sigma_0|^{1/2}}
                  \exp\left(-\frac{1}{2}(x_{new} - \mu_0)^T \Sigma_0^{-1}(x_{new} - \mu_0)\right) \\
                   & = \cdots = 3.47826 \cdot 10^{-48}                                            \\
                  P(x_{new} \mid z = 1) \sim \mathcal{N}(x_{new} \mid \mu_1, \Sigma_1)
                   & = \frac{1}{(2\pi)^{2/2}|\Sigma_1|^{1/2}}
                  \exp\left(-\frac{1}{2}(x_{new} - \mu_1)^T \Sigma_1^{-1}(x_{new} - \mu_1)\right) \\
                   & = \cdots = 0.000107642
                \end{align*}
                }

                With $z=1$ having a higher likelihood (and thus a higher posterior probability),
                the query vector $x_{new}$ is most likely to belong to class $z=1$.
          \item {
                Modelling the likelihoods as 1-dimensional Gaussians (instead of the 2-dimensional
                ones utilized in the previous exercise) ends up being a simplification of the
                problem; we can reutilize the previously computed mean vectors
                to compute the likelihoods for each class, as follows:

                $$
                  \mu_0 = \begin{bmatrix}
                    \mu_0^{(1)} \\
                    \mu_0^{(2)}
                  \end{bmatrix} = \begin{bmatrix}
  93.3333\\
  156.667\\
\end{bmatrix}, \quad
                  \mu_1 = \begin{bmatrix}
                    \mu_1^{(1)} \\
                    \mu_1^{(2)}
                  \end{bmatrix} = \begin{bmatrix}
  80\\
  203.333\\
\end{bmatrix}
                $$
                $$
                  \sigma_0 = \begin{bmatrix}
                    \sqrt{\Sigma_0^{(1, 1)}} \\
                    \sqrt{\Sigma_0^{(2, 2)}}
                  \end{bmatrix} = \begin{bmatrix}
  66.5833\\
  5.7735\\
\end{bmatrix}, \quad
                  \sigma_1 = \begin{bmatrix}
                    \sqrt{\Sigma_1^{(1, 1)}} \\
                    \sqrt{\Sigma_1^{(2, 2)}}
                  \end{bmatrix} = \begin{bmatrix}
  10\\
  15.2753\\
\end{bmatrix}
                $$

                Note how, since we're now working with conditionally independent variables,
                we abandon the notion of a covariance matrix (and, as such, the covariance
                between all features is zero), and instead work with the variances of each
                feature.
                Here, we're going to split the likelihoods, utilizing the Naive Bayes assumption,
                into two 1-dimensional Gaussians, one for each feature:

                $$
                  P(x \mid z = c) \sim
                  \mathcal{N}(\boldsymbol{x} \mid \boldsymbol{\mu_c}, \boldsymbol{\sigma_c}) =
                  \prod_{i=1}^d \frac{1}{\sqrt{2\pi\sigma_{c,i}^2}}
                  \exp\left(-\frac{1}{2\sigma_{c,i}^2}(x_i - \mu_{c,i})^2\right)
                $$

                Once again, we only need to know the likelihoods in order to decide
                which class the query vector $x_{new}$ belongs to, since the priors
                and the denominator are the same for both classes. The likelihoods
                for each class are the following (once again, intermediate steps
                aren't shown for brevity):

                \begin{align*}
                  P(x_{new} \mid z = 0) \sim \mathcal{N}(x_{new} \mid \mu_0, \sigma_0)
                  = \prod_{i=1}^2 \frac{1}{\sqrt{2\pi\sigma_{0,i}^2}}
                  \exp\left(-\frac{1}{2\sigma_{0,i}^2}(x_{new,i} - \mu_{0,i})^2\right)
                  = \cdots = 1.57083 \cdot 10^{-34} \\
                  P(x_{new} \mid z = 1) \sim \mathcal{N}(x_{new} \mid \mu_1, \sigma_1)
                  = \prod_{i=1}^2 \frac{1}{\sqrt{2\pi\sigma_{1,i}^2}}
                  \exp\left(-\frac{1}{2\sigma_{1,i}^2}(x_{new,i} - \mu_{1,i})^2\right)
                  = \cdots = 5.1566 \cdot 10^{-5}
                \end{align*}

                We opt to assign the query vector to class $z=1$, since it has a higher likelihood.
                }
        \end{enumerate}

        \question {
  \item Assuming training examples with $m$ features and a binary class:

        \begin{enumerate}
          \item How many parameters are needed to estimate, considering boolean features and:
                \begin{enumerate}
                  \item No assumptions regarding the data's distribution.
                  \item Naive Bayes assumption.
                \end{enumerate}
          \item How many parameters are needed to estimate, considering numeric-valued features and:
                \begin{enumerate}
                  \item Multivariate Gaussian assumption.
                  \item Naive Bayes with a Gaussian assumption.
                \end{enumerate}
        \end{enumerate}
        }

        For starters, it's worth noting that we'll always need to estimate at least
        one prior, independently of the features' type and the assumptions we make
        regarding the data. Considering a strictly binary class, we'll only need
        to estimate one prior, be it $\pi_k$, since the other one is trivially $1 - \pi_k$.

        \begin{enumerate}
          \item {
                \begin{enumerate}
                  \item {
                        Working with the classic Bayesian model, we'll need to estimate
                        the likelihoods for each class; since we're working with boolean
                        features, we'll need to estimate the probability of each feature
                        being true/false for each class. There are, of course, $2^m$ possible
                        combinations of boolean features for each class (since we can't assume
                        any type of conditional independence between features); however, since
                        the sum of the probabilities of all possible combinations must
                        be equal to 1, we can reduce the number of parameters to be estimated
                        to $2^m - 1$. As such, we'll need to estimate $2(2^m - 1) + 1$ parameters
                        in total, since we'll also need to estimate the prior for each class.
                        }
                  \item {
                        Working with the Naive Bayes assumption - i.e, assuming that the features
                        are conditionally independent given the class - we'll need to estimate
                        the probability of each feature being true/false for each class;
                        this way, there'll be $2m$ possible features combinations, and as such,
                        we'll need to estimate $2m + 1$ parameters in total!
                        }
                \end{enumerate}
                }
          \item {
                \begin{enumerate}
                  \item {
                        Working with a multivariate Gaussian distribution, we'll need to estimate
                        the mean vector and the covariance matrix for each class; while we need
                        to estimate all $m$ parameters for the mean vector (per class), we can utilize the
                        fact that the covariance matrix is symmetric to reduce the number of
                        parameters to be estimated to $\frac{m(m+1)}{2}$ per class, the amount of entries
                        in the upper triangular part of an $m \times m$ matrix. As such, we'll
                        need to estimate $2(m + \frac{m(m+1)}{2}) + 1$ parameters in total.
                        }
                  \item {
                        Working with the Naive Bayes gaussian assumption, we'll need to estimate
                        the mean vector and the variance vector for each class; here, since
                        there are no covariance parameters to be estimated, we'll need to
                        estimate $m$ parameters for the mean vector and $m$ parameters for the
                        variance vector, per class. As such, we'll need to estimate $2(2m) + 1$
                        parameters in total.
                        }
                \end{enumerate}
                }
        \end{enumerate}

\end{enumerate}

\end{document}