\documentclass[12pt]{article}
\usepackage[paper=letterpaper,margin=1.5cm]{geometry}
\usepackage{amsmath}
\usepackage{amssymb}
\usepackage{amsfonts}
\usepackage{newtxtext, newtxmath}
\usepackage{enumitem}
\usepackage{titling}
\usepackage{svg}
\usepackage{xcolor}
\usepackage{listings}
\usepackage{float}
\usepackage{nicefrac}
\usepackage[most]{tcolorbox}
\usepackage[colorlinks=true]{hyperref}

\setlength{\droptitle}{-6em}

\definecolor{codegreen}{rgb}{0,0.6,0}
\definecolor{codegray}{rgb}{0.5,0.5,0.5}
\definecolor{codepurple}{rgb}{0.58,0,0.82}
\definecolor{backcolour}{rgb}{0.95,0.95,0.92}
\definecolor{bg}{rgb}{1,0.96,0.9}

\lstdefinestyle{mystyle}{
  commentstyle=\color{codegreen},
  keywordstyle=\color{magenta},
  numberstyle=\tiny\color{codegray},
  stringstyle=\color{codepurple},
  basicstyle=\ttfamily\footnotesize,
  breakatwhitespace=false,
  breaklines=true,
  captionpos=b,
  keepspaces=true,
  numbers=left,
  numbersep=5pt,
  showspaces=false,
  showstringspaces=false,
  showtabs=false,
  tabsize=2
}

% \tcbset{enlarge left by=-0.8cm,left=1.2cm,enlarge right by=-2cm,right=0.8cm}

\lstset{
  style=mystyle,
  inputencoding=utf8,
  extendedchars=true,
}

\begin{document}

The perceptron algorithm consists of one of the simplest types of neural
network architectures. It is a single-layer neural network with a single
neuron. The neuron is a linear combination of the input variables, with a
bias term, and then passed through an activation function. The activation
function is used to determine the output of the neuron.
The algorithm itself is a supervised learning algorithm, meaning that it
requires labeled training data to train the model. The algorithm is
iterative, hence it'll continue to update the weights until the
model converges. The learning rate is a hyperparameter that controls the
step size of the weight updates. The activation function is a hyperparameter
that determines the output of the neuron. The learning rule is as follows:

\begin{equation*}
  w^{new} \leftarrow w^{old} + \eta \cdot (z - \hat{z}(x)) \cdot x, \quad
  \hat{z}(x) = f(net(x)), \quad
  net(x) = \sum_{i=1}^n w_i \cdot x^{(i)} + w_0
\end{equation*}

where $w_i$ is the weight for the $i$th input variable, $\eta$ is the
learning rate, $z$ is the true label, $\hat{z}$ is the predicted label,
and $x_i$ is the $i$th input variable. The learning rule is applied for
each training example, and an epoch is a single pass through the entire
training set.

\begin{enumerate}[leftmargin=\labelsep]
  \begin{tcolorbox}[enhanced jigsaw,colback=bg,boxrule=0pt,arc=1pt,halign=center]
    \item Considering the following linearly separable training data:

    \begin{table}[H]
      \centering
      \begin{tabular}{c|c|c|c|c}
              & $y_1$ & $y_2$ & $y_3$ & $z$ \\ \hline
        $x_1$ & 0     & 0     & 0     & -1  \\
        $x_2$ & 0     & 2     & 1     & 1   \\
        $x_3$ & 1     & 1     & 1     & 1   \\
        $x_4$ & 1     & -1    & 0     & -1
      \end{tabular}
    \end{table}

    Given the perceptron learning algorithm with a learning rate $\eta = 1$,
    sign activation and all weights initialized to one (including the bias):

    \begin{enumerate}
      \item Considering $y_1$ and $y_2$, apply the algorithm until convergence.
            Draw the separation hyperplane.
      \item Considering all input variables, apply one epoch of the algorithm.
            Do weights change for an additional epoch?
      \item Identify the perceptron output for $x_{new}$ = $\begin{bmatrix} 0 & 0 & 1 \end{bmatrix}^T$.
      \item What happens if we replace the sign function with the step function? Specifically,
            how would you change $\eta$ to ensure the same results?
    \end{enumerate}
  \end{tcolorbox}

  \begin{enumerate}
    \item {
      As per the question statement, we're working with $\eta = 1$ and sign activation:

      $$
      \hat{z} = f(net) = \begin{cases}
        1 & net \geq 0 \\
        -1 & net < 0
      \end{cases}
      $$
      
      Moreover, we're considering weights (and the bias, $w_0$) initialized at one.
      Performing epochs following the $\{x_1, \cdots, x_4\}$ order, we get the following
      weight updates:

      \begin{align*}
        \hat{z}(x_1) &= f(net(x_1)) = f(1 + 1 \cdot 0 + 1 \cdot 0) = f(1) = 1 \\
        \hat{z}(x_2) &= f(net(x_2)) = f(1 + 1 \cdot 0 + 1 \cdot 2) = f(3) = 1 \\
        \hat{z}(x_3) &= f(net(x_3)) = f(1 + 1 \cdot 1 + 1 \cdot 1) = f(3) = 1 \\
        \hat{z}(x_4) &= f(net(x_4)) = f(1 + 1 \cdot 1 + 1 \cdot -1) = f(1) = 1
      \end{align*}

      \begin{align*}
        x_1: \quad & w \leftarrow \begin{bmatrix}
  0.1 & 0.1 & 0.1 & 0.1 & 0.1\\
  0.1 & 0.1 & 0.1 & 0.1 & 0.1\\
  0.1 & 0.1 & 0.1 & 0.1 & 0.1\\
\end{bmatrix} + 1 \cdot (-1 - 1) \cdot \input{aux-matrices/ex-1/x_1_first_3} = \begin{bmatrix}
  -0.926984\\
  -2.90718\\
  -4.91178\\
\end{bmatrix} \\
        x_2: \quad & w \leftarrow \begin{bmatrix}
  -0.926984\\
  -2.90718\\
  -4.91178\\
\end{bmatrix} + 1 \cdot (1 - 1) \cdot \begin{bmatrix}
  1 & 0 & 2\\
\end{bmatrix} = \input{aux-matrices/ex-1/w_3} \\
        x_3: \quad & w \leftarrow \input{aux-matrices/ex-1/w_3} + 1 \cdot (1 - 1) \cdot \input{aux-matrices/ex-1/x_3_first_3} = \input{aux-matrices/ex-1/w_4} \\
        x_4: \quad & w \leftarrow \input{aux-matrices/ex-1/w_4} + 1 \cdot (-1 - 1) \cdot \input{aux-matrices/ex-1/x_4_first_3} = \input{aux-matrices/ex-1/w_5}
      \end{align*}

      After entering a new epoch, we'd update the weights again, this time for $x_1$;
      since such an update wouldn't lead to an actual update on the weight matrix,
      that'd make a full pass on the training set without changes, and, as such,
      the algorithm would converge. Therefore, the regression hyperplane is given by:

      $$
        -1 + x_1 + x_2 = 0 \leftrightarrow x_2 = 1 - x_1
      $$

      % TODO: Plotting the data and the hyperplane, we get the following:
    }
    \item {
      Here, we apply the same algorithm as before, but now considering all input variables.

      \begin{align*}
        \hat{z}(x_1) &= f(net(x_1)) = f(1 + 1 \cdot 0 + 1 \cdot 0 + 1 \cdot 0) = f(1) = 1 \\
        \hat{z}(x_2) &= f(net(x_2)) = f(1 + 1 \cdot 0 + 1 \cdot 2 + 1 \cdot 1) = f(4) = 1 \\
        \hat{z}(x_3) &= f(net(x_3)) = f(1 + 1 \cdot 1 + 1 \cdot 1 + 1 \cdot 1) = f(4) = 1 \\
        \hat{z}(x_4) &= f(net(x_4)) = f(1 + 1 \cdot 1 + 1 \cdot -1 + 1 \cdot 0) = f(2) = 1
      \end{align*}

      \begin{align*}
        x_1: \quad & w \leftarrow \input{aux-matrices/ex-1/w_1_all} + 1 \cdot (-1 - 1) \cdot \begin{bmatrix}
  0.197375 & 0.761594\\
  0.197375 & 0.761594\\
  0.197375 & 0.761594\\
  0.197375 & 0.761594\\
\end{bmatrix} = \input{aux-matrices/ex-1/w_2_all} \\
        x_2: \quad & w \leftarrow \input{aux-matrices/ex-1/w_2_all} + 1 \cdot (1 - 1) \cdot \begin{bmatrix}
  1 & 0 & 2 & 1\\
\end{bmatrix} = \input{aux-matrices/ex-1/w_3_all} \\
        x_3: \quad & w \leftarrow \input{aux-matrices/ex-1/w_3_all} + 1 \cdot (1 - 1) \cdot \begin{bmatrix}
  0.162507 & 0.215688\\
  0.162507 & 0.215688\\
  0.162507 & 0.215688\\
\end{bmatrix} = \input{aux-matrices/ex-1/w_4_all} \\
        x_4: \quad & w \leftarrow \input{aux-matrices/ex-1/w_4_all} + 1 \cdot (-1 - 1) \cdot \begin{bmatrix}
  1\\
  3\\
  3\\
\end{bmatrix} = \input{aux-matrices/ex-1/w_5_all}
      \end{align*}

      Once again, we'd enter a new epoch and update the weights considering the first sample,
      but since such an update wouldn't lead to an actual update on the weight matrix,
      that'd make a full pass on the training set without changes, and, as such,
      the algorithm would converge. \textbf{An additional epoch wouldn't, therefore,
      change the weights}.
    }
    \item {
      As we've mentioned before, the perceptron's output is given by the
      activation function - here, the sign of the net input. Therefore, the
      perceptron's output is given by:

      $$
      \hat{z} = f(net) = \begin{cases}
        1 & net \geq 0 \\
        -1 & net < 0
      \end{cases}
      $$

      Here, considering the weights computed in the previous question, we'll
      have the following:

      $$
        net(x_{new}) = -1 + 1 \cdot 0 + 1 \cdot 0 + 1 \cdot 1 = 0
      $$

      As we know, $f(0) = 1$, and, as such, the perceptron would classify the
      new sample as belonging to the binary class $1$.
    }
    \item {
      As we know, the step function is rather similar to the sign function,
      defined as follows:

      $$
        f(net) = \begin{cases}
          1 & net \geq 0 \\
          0 & net < 0
        \end{cases}
      $$

      Let's consider the following learning rules (the first one considers
      the activation function as the sign function, and the second one
      considers the activation function as the step function):

      \begin{align*}
        \text{Sign function:} \quad & w \leftarrow w + \eta \cdot (z - sign(net)) \cdot x \\
        \text{Step function:} \quad & w \leftarrow w + \eta \cdot (z - step(net)) \cdot x
      \end{align*}

      We can note, of course, that the only differing term between both rules
      is $(z - \hat{z})$; therefore, to make it so that both rules are
      equivalent, we'd have to make sure that the step function's output
      is correctly balanced with the sign function's output, utilizing a scalar
      factor (be it $\tau$) for such purpose, effectively altering the learning
      rate to be $\eta_{sign} = \tau \cdot \eta_{step}$. Let's try to find $\tau$:

      $$
      z - sign(net) = \tau (z - step(net))
      $$

      We know that $z \in \{-1, 1\}$, and, as such, we can write each side of
      the equation's interval as follows:

      $$
      [-1, 1] - [-1, 1] = \tau ([-1, 1] - [0, 1]) \leftrightarrow
      [-2, 2] = \tau [-1, 1]
      $$

      As such, if we want an equal learning rule utilizing both step and sign
      functions, we'd have to set $\tau = 2$, and, as such, get the following
      learning rule:

      $$
        w \leftarrow w + 2 \eta_{sign} \cdot (z - step(net)) \cdot x \quad \leftrightarrow \quad
        w \leftarrow w + \eta_{sign} \cdot (z - sign(net)) \cdot x
      $$
    }
  \end{enumerate}

  \begin{tcolorbox}[enhanced jigsaw,colback=bg,boxrule=0pt,arc=1pt,halign=center]
    \item Let us consider the following activation function:

    \begin{equation*}
      \hat{z}(x, w) = \frac{1}{1 + e^{-2 w x}}
    \end{equation*}

    Consider also the half sum of squared errors as the loss function:

    \begin{equation*}
      E(w) = \nicefrac{1}{2} \sum_{i=1}^N (z_i - \hat{z}(x_i, w))^2
    \end{equation*}

    \begin{enumerate}
      \item Determine the gradient descent learning rule for this unit.
      \item Compute the first gradient descent update, assuming an initialization of all ones.
      \item Compute the first stochastic gradient descent update assuming an initialization of all ones.
    \end{enumerate}
  \end{tcolorbox}

  \begin{enumerate}
    \item {}
    \item {}
    \item {}
  \end{enumerate}

  \begin{tcolorbox}[enhanced jigsaw,colback=bg,boxrule=0pt,arc=1pt,halign=center]
    \item Let us consider the following activation function:

    \begin{equation*}
      \hat{z}(x, w) = \frac{1}{1 + e^{-w x}}
    \end{equation*}

    Here, we'll be using the cross-entropy loss function:

    \begin{equation*}
      E(w) = - \sum_{i=1}^N z_i \log \hat{z}(x_i, w) + (1 - z_i) \log (1 - \hat{z}(x_i, w))
    \end{equation*}

    \begin{enumerate}
      \item Determine the gradient descent learning rule for this unit.
      \item Compute the first gradient descent update, assuming an initialization of all ones.
      \item Compute the first stochastic gradient descent update assuming an initialization of all ones.
    \end{enumerate}
  \end{tcolorbox}

  \begin{enumerate}
    \item {}
    \item {}
    \item {}
  \end{enumerate}

  \begin{tcolorbox}[enhanced jigsaw,colback=bg,boxrule=0pt,arc=1pt,halign=center]
    \item Consider now the activation function described in the previous exercise,
    paired with the half sum of squared errors loss function.

    \begin{enumerate}
      \item Determine the gradient descent learning rule for this unit.
      \item Compute the stochastic gradient descent update for input $x_{new}$ = $\begin{bmatrix} 1 & 1 \end{bmatrix}^T$,
            $z_{new} = 0$, with initial weights $w = \begin{bmatrix} 0 & 1 & 0 \end{bmatrix}^T$
            and learning rate $\eta = 2$.
    \end{enumerate}
  \end{tcolorbox}

  \begin{enumerate}
    \item {}
    \item {}
  \end{enumerate}

  \begin{tcolorbox}[enhanced jigsaw,colback=bg,boxrule=0pt,arc=1pt,halign=center]
    \item Consider the sum squared and cross-entropy loss functions. Any stands out?
    What changes when one changes the loss function?
  \end{tcolorbox}

\end{enumerate}

\end{document}